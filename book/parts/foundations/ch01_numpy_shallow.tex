
\chapter{NumPy Basics}

\section*{Goals}
\begin{itemize}
  \item Understand NumPy arrays, dtypes, and shapes
  \item Indexing \& slicing (1D/2D)
  \item Reshape, view vs copy
  \item Broadcasting
  \item Vectorization vs loops
  \item Descriptive statistics \& linear algebra
  \item Random numbers with reproducibility
  \item Mini CSV task
\end{itemize}

\section{Environment \& Version}
Always check library version for reproducibility.
\begin{lstlisting}[language=Python]
import numpy as np
print("NumPy version:", np.__version__)
\end{lstlisting}

\section{Arrays, dtypes, and shapes}
Arrays are the core structure in NumPy. They are contiguous in memory and efficient.

\begin{lstlisting}[language=Python]
a = np.array([1,2,3,4,5])
print("a:", a, "| shape:", a.shape, "| ndim:", a.ndim, "| dtype:", a.dtype)

b = np.array([[1,2,3],[4,5,6]])
print("b:\n", b, "\nshape:", b.shape, "| ndim:", b.ndim, "| dtype:", b.dtype)

zeros = np.zeros((2,3))
ones = np.ones((2,3))
rng = np.arange(0,10,2)
lin = np.linspace(0,1,5)
\end{lstlisting}

\section{Indexing \& Slicing}
Indexing extracts elements. In 2D arrays: [row, column].
\begin{lstlisting}[language=Python]
c = np.arange(10)
print("c:", c)
print("c[2:7]:", c[2:7])
print("b[0,1]:", b[0,1])
print("b[:,1]:", b[:,1])
\end{lstlisting}

\section{Reshape, Views, and Copies}
Reshape changes view. Copy creates independence.
\begin{lstlisting}[language=Python]
d = np.arange(12)
m = d.reshape(3,4)
m[0,0] = 999
print("d[0]:", d[0])

m_copy = m.copy()
m_copy[0,0] = -1
\end{lstlisting}

\section{Broadcasting}
Broadcasting expands smaller arrays automatically.
\begin{lstlisting}[language=Python]
X = np.ones((3,3))
v = np.array([1,2,3])
print(X + v)

A = np.arange(6).reshape(2,3)
B = np.arange(2).reshape(2,1)
print(A + B)
\end{lstlisting}

\section{Vectorization vs Loops}
Vectorization uses C-level speed. Loops are slow.
\begin{lstlisting}[language=Python]
x = np.arange(100_000)
y_vec = x*x + 2*x + 1

def poly_slow(xx):
    out = []
    for t in xx:
        out.append(t*t + 2*t + 1)
    return np.array(out)

y_slow = poly_slow(x)
\end{lstlisting}

\section{Descriptive Statistics}
\begin{lstlisting}[language=Python]
arr = np.array([3,7,2,9,5,10,4])
print("mean:", arr.mean(), "| std:", arr.std())
print("percentiles:", np.percentile(arr, [25,50,75]))
\end{lstlisting}

\section{Linear Algebra}
\begin{lstlisting}[language=Python]
M = np.array([[2,0],[0,3]])
v = np.array([5,1])
print("M @ v =", M @ v)
print("det(M) =", np.linalg.det(M))
print("inv(M) =", np.linalg.inv(M))
\end{lstlisting}

\section{Random Numbers}
\begin{lstlisting}[language=Python]
rng = np.random.default_rng(seed=42)
print(rng.normal(0,1,5))
print(rng.integers(0,10,5))
\end{lstlisting}

\section{Mini Task: CSV}
\begin{lstlisting}[language=Python]
import os, csv
os.makedirs("data", exist_ok=True)
csv_path = os.path.join("data","toy.csv")
with open(csv_path,"w",newline="") as f:
    w = csv.writer(f)
    w.writerow(["id","value"])
    w.writerows([[1,10],[2,20],[3,35],[4,50]])

data = np.genfromtxt(csv_path, delimiter=",", names=True)
vals = data["value"]
print("mean:", np.mean(vals), "| sum:", np.sum(vals))
\end{lstlisting}

\section*{Summary}
\begin{itemize}
  \item Arrays = vectors/matrices/tensors
  \item Shapes, dtypes = structure of data
  \item Indexing = select subsets
  \item Reshape = flexible views; Copy = independence
  \item Broadcasting = smart expansion
  \item Vectorization = speed via array ops
  \item Statistics + Linear Algebra = math backbone
  \item Random numbers = reproducibility
  \item CSV = dataset handling
\end{itemize}
